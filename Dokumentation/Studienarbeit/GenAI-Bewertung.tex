\documentclass[a4paper,12pt]{article}
\usepackage[utf8]{inputenc}
\usepackage[ngerman]{babel}
\usepackage[T1]{fontenc}
\usepackage[scaled]{uarial}
\usepackage{amsmath}
\usepackage{titling}
\usepackage{fancyhdr}
\usepackage{calc}
\usepackage[a4paper, margin=2.5cm]{geometry}
\usepackage{tocloft}
\usepackage{graphicx}
\usepackage{caption}
\usepackage{acronym}
\usepackage{upquote}
\usepackage[style=authoryear,backend=biber]{biblatex}
\addbibresource{Literature.bib}
\usepackage{csquotes}
\usepackage{filecontents}
\usepackage{setspace}
\usepackage{ragged2e}
\usepackage[backend=biber]{biblatex}
\usepackage{framed}
\usepackage{tabularx}
\usepackage{geometry}
\usepackage{fancyhdr}
\usepackage{enumitem}
\usepackage[parfill]{parskip}

\setlength\bibnamesep{1.5\itemsep}
\renewcommand{\cftsecleader}{\cftdotfill{\cftdotsep}}
\linespread{1.5}
\pagestyle{fancy}
\renewcommand*\familydefault{\sfdefault}
\fancyhf{}
\setlength{\parskip}{6pt}
\onehalfspacing
\justifying
\fancyhead[L]{Entwicklung einer Administrativen Applikation für eine Swagger UI}
\cfoot{\tiny Praxisbericht / Stand \today \hfill \thepage}
\title{Einsatz von GenAI-Technologien zur Bewertung von Antworten}
\author{Len Vejsada}
\date{\today}
\geometry{a4paper, margin=1in}
\setlength{\parindent}{0pt}


\renewcommand{\maketitle}{
  \begin{center}
    {\LARGE\textbf{\thetitle}}\\[2em]
    {{PRAXISBERICHT\\[2em]
     des Studiengangs Informatik\\[0.5em]
     an der \\[0.5em]
     Dualen Hochschule Baden-Württemberg Karlsruhe\\[0.5em]
     von}}\\[1em]
    \theauthor\\[1em]
    \thedate\\[7em]
  \end{center}
}

\fancypagestyle{firstpage}{
  \fancyhf{}
  \renewcommand{\headrulewidth}{0pt}
  \cfoot{\tiny Praxisbericht / Stand \today \hfill \thepage}
}


\begin{document}
\thispagestyle{firstpage}
\begin{figure}
\begin{center}
  \includegraphics[width=\textwidth]{Bilder/KardexRemstar.png}
  \label{fig:logo}
\end{center}
\end{figure}
  
\maketitle

\begin{tabular}{l@{\hspace{2em}}l}
  Matrikelnummer & 9447853 \\[0.5em]
  Kurs & TINF22B6 \\[0.5em]
  Ausbildungsfirma & Kardex, Bellheim \\[0.5em]
  Betreuer & Simon Franke \\[0.5em]
\end{tabular}

\newpage


\thispagestyle{firstpage}
\textbf{Erklärung} \\
(gemäß §5(3) der „Studien- und Prüfungsordnung DHBW Technik“ vom
29. 9. 2015)
Ich versichere hiermit, dass ich meinen Praxisbericht mit dem Thema: „Einsatz von GenAI-Technologien zur Bewertung von Antworten“ selbstständig verfasst und keine anderen als die angegebenen Quellen und Hilfsmittel benutzt habe. Ich versichere zudem, dass die eingereichte elektronische Fassung mit der gedruckten Fassung übereinstimmt.\\[1.5em]
\vspace{1cm}

\begin{center}
\begin{tabularx}{\textwidth}{X r}
Lustadt, 12.12.2024 & gez. Len Vejsada \\
\hline
\end{tabularx}
\end{center}

\noindent {\small Ort, Datum \hfill Unterschrift}\\

\newpage
\tableofcontents
\thispagestyle{fancy}
\thispagestyle{firstpage}
\newpage

\addcontentsline{toc}{section}{Abbildungsverzeichnis}
\listoffigures

\newpage
\section*{Abkürzungsverzeichnis}
\addcontentsline{toc}{section}{Abkürzungsverzeichnis}
\begin{acronym}
  \acro{CSharp}[C\#]{C Sharp}
  \acro{HTML}{Hypertext Markup Language}
  \acro{CSS}{Cascading Style Sheets}
  \acro{SQL}{Structured Query Language}
  \acro{JS}{JavaScript}
  \acro{.NET}{Domain Network}
  \acro{ASP.NET}{Active Server Pages .NET}
  \acro{EF}{Entity Framework}
  \acro{DDD}{Domain Driven Design}
  \acro{ORM}{Object-Relational Mapping}
  \acro{API}{Application Programming Interface}
  \acro{SPA}{Single-Page Application}
  \acro{MPA}{Multi-Page Application}
  \acro{UI}{User Interface}
  \acro{DI}{Dependency Injection}
  \acro{TS}{TypeScript}
  \acro{CQRS}{Command Query Responsibility Segregation}
  \acro{IoC}{Inversion of Control}
  \acro{SOA}{Service-Oriented Architecture}
  \acro{CLI}{Command Line Interface}
  \acro{TFVC}{Team Foundation Version Control}
  \acro{CI}{Continuous Integration}
  \acro{CD}{Continuous Deployment}
  \acro{YAML}{Yet Another markup Language}
  \acro{REST}{Representational State Transfer}

\end{acronym}

\newpage
\section{Einleitung}
\subsection{Ziele des Dokuments}
\subsection{Anlass und Motivation des Projekts}
\subsection{Aufbau der Arbeit}

\newpage

\section{Hintergrund und Kontext}
\subsection{Technischer Hintergrund}
\subsection{Generative KI-Technologien: Definition und Bedeutung}
\subsection{Vergleich von KI- und klassischen Algorithmen in der Bewertung}

\newpage

\section{Projektbeschreibung}
\subsection{Projektziele}
\subsection{Überblick über das Minimum Viable Product (MVP)}
\subsection{Abgrenzung und Einschränkungen}

\newpage


\section{Technologien und Architektur}
\subsection{Frontend}
\subsection{Backend}
\subsection{Datenbank}
\subsection{API-Kommunikation}
\subsection{Teststrategie}

\newpage

\section{Implementierung}
\subsection{Implementierung der Fragetypen}
\subsection{Integration der generativen KI-Bewertung}
\subsection{Datenbankstruktur und Datenmodelle}
\subsection{Entwicklung der Benutzeroberfläche}

\newpage

\section{Evaluierung}
\subsection{Vergleich von GenAI-Lösungen}
\subsection{Leistung und Genauigkeit der Bewertung}
\subsection{Kostenanalyse der verschiedenen KI-APIs}

\newpage

\section{Ergebnisse und Diskussion}
\subsection{Zusammenfassung der Implementierungsergebnisse}
\subsection{Herausforderungen bei der Entwicklung}
\subsection{Potenzial für zukünftige Erweiterungen}

\newpage

\section{Zukünftige Arbeiten und Erweiterungsmöglichkeiten}
\subsection{Adaptive Lernpfade}
\subsection{Integration von Sprach- und Bilderkennung}
\subsection{Mehrsprachigkeit}

\newpage

\section{Fazit}
\subsection{Erfüllung der Projektziele}
\subsection{Bedeutung der Ergebnisse}
\subsection{Ausblick}


\newpage
\nocite{*}
\printbibliography


\end{document}
